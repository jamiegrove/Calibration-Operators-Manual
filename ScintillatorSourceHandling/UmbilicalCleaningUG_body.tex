
\fancyhf{}
% \lhead{\begin{tabular}{|p{8.25cm}} \hline {\large \bf SNO+ Laser Procedures} \\ \\ \\ \\ \end{tabular}}
\lhead{\begin{tabular}{|p{8.25cm}|p{3cm}|p{2.5cm}|p{1.5cm}|}
                \hline
            {\large \bf SNO+ Umbilical Cleaning}
                     & Document No:
                     & Revision No: 01
              %                                                  & Effective Date: 2022-03-21\\
                                                   & Page \thepage \\
\hline
                                                     \end{tabular}}


\begin{tabular}{||l|l|l||}
\hline\hline
& \multicolumn{2}{p{8cm}||}{\bf SNO+ Umbilical Cleaning: Underground} \\
\includegraphics[width=6cm]{figures/SNOplus_logo.png} & \multicolumn{2}{p{8cm}||}{} \\
\hline
\multicolumn{2}{||p{8.5cm}|}{Document Number:} & Revision Number: 01\\
\hline
\multicolumn{3}{||l||}{Document Owner: SNO+ Calibration Post-Doc} \\
\hline
\multicolumn{3}{||l||}{Reviewer:}\\
\hline
Name: & Signature & Date \\
\hline
\multicolumn{3}{||l||}{Authorizer:}\\
\hline
Name: & Signature & Date \\
\hline\hline
\end{tabular}
\thispagestyle{empty}

\newcounter{cbcounter}
\newcommand\showcbcounter{\stepcounter{cbcounter}\thecbcounter}

\section{Requirements}

\begin{itemize}
\item Umbilical underground (containing fibre bundle)
\item Source of LAB (from 60 tonne tanks)
\item Cleaning vessel (aluminum with sealed lid and two nozzles; one terminates to a three way valve with a check valve on one side, second terminate to a standard bellows valve and extends into the cleaning volume through a steel pipe). Must be large enough to contain 30 m of 5/8'' umbilical. Suspect 50 cm diameter, 50 cm tall will be sufficient.
\item Spill kit available during movement of LAB
\item Extra cleaning pads for removing LAB after soaking is complete
\item Gloves must be worn at all times during handling of umbilical and volume
\item Nitrogen source for preparing vessel and to maintain blanket during soaking
\item Radon proof bag to store the umbilica after soaking
\end{itemize}

\section{Procedure}
\subsection{Starting the umbilical soak}
\begin{enumerate}
\item Prepare the cleaning vessel
  \begin{enumerate}
  \item In the chem lab, wipe the vessel inside and out with UPW. 
  \item Connect the lid of the vessel. Purge the vessel with nitrogen.
  \end{enumerate}
\item Open the Umbilical bag and vessel lid. Quickly put the umbilical into the vessel.
\item Close the vessel lid and resume nitrogen purge.
\item Put the vessel on a cart and transport the vessel to the 60 tonne tank area.
\item With two or more people lift the vessel into the 60 tonne tank area and put it in the corner next to the stairs
\item Fill the vessel with LAB through the standard bellows valve (same as the valve used for the nitrogen).
\item Leave room to create a nitrogen blanket within the vessel. Add further nitrogen as necessary. (If necessary create a tent around the lid and fill that with nitrogen).
\item Allow the umbilical to sit for approximately 35 days.
\end{enumerate}  

\subsection{Concluding the umbilical soak}
\begin{enumerate}
\item Connect nitrogen to the three way valve oposite to the checkvalve. Connect the bellows valve to a waste tote via a peristalic pump.
\item Syphon off the LAB while adding nitrogen to keep air from getting into the space. Once the flow stops or becomes intermittant, close the valves to the vessel, stop the pump, and disconnect the syphon line and nitrogen. 
\item Lift the cleaning vessel out of the 60 tonne tank area. Place it on a cart and transport it to the chem lab.
\item Open the vessel. Wipe off residual LAB with cleaning pad while transferring umbilical to radon proof bag. 
\item Fill the bag with nitrogen and seal the bag once transfer is complete.
\item Take umbilical to DCR to wait for installation on URM.
\end{enumerate}
